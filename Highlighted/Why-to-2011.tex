Data for our empirical measure of credit conditions are available starting 1966q2, and the data we use in estimation cover that date to 2011q4.

We do not use data after 2011 for several reasons. First, personal saving rate statistics are subject to large revisions until some five years  after the first data release (after the BEA receives much higher quality personal income data from the IRS).  To quote \cite{nsSavingRevisions}: ``[M]uch of the initial variation in the personal saving rate across time was meaningless noise.''\footnote{In addition, there were substantial gyrations in the saving rate in 2012 due to a tax-related anomaly (in late 2012 income was boosted by accelerated and special dividend payments and by accelerated bonus payments in anticipation of changes in individual income tax rates in 2013).} As  their paper documents, it is not uncommon for the saving rate to be revised by 3--5 percent of disposable income after several benchmark revisions.

Second, our index of credit availability is increasingly questionable after 2011 because of the apparent divergence in credit conditions for installment and mortgage loans: Various sources (including the Mortgage Credit Availability Index of the Mortgage Bankers' Association; see also \cite{bhutta_mortgageDebt}) document continued tight credit conditions after 2011.  If, as this work suggests, mortgage credit remained tighter than indicated by our installment loans index after 2012, that could explain part of the continued high saving rate in the post-2012 period, which would be mispredicted by a mismeasured credit conditions index.  Alternatively, saving attitudes may have changed after the Great Recession due to the substantial shock of the Great Recession, perhaps because of ``scarring'' effects  (see, e.g., \cite{malmendierSheng}, \cite{jstLeveragedBubbles}, \cite{hallQuantifying}); for evidence that financial crises have much longer-lasting effects than usual business cycle fluctuations, see \cite{rrAftermath}.  All of these are reasons to worry about data from the post-Great-Recession period.
