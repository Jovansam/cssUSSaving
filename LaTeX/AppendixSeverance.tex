\providecommand{\econtexRoot}{..}
% The \commands below are required to allow sharing of the same base code via Github between TeXLive on a local machine and ShareLaTeX.  This is an ugly solution to the requirement that custom LaTeX packages be accessible, and that ShareLaTeX seems to ignore symbolic links (even if they are relative links to valid locations)
\providecommand{\econtex}{./texmf-local/tex/latex/econtex}
\providecommand{\econtexSetup}{./texmf-local/tex/latex/econtexSetup}
\providecommand{\econtexShortcuts}{./texmf-local/tex/latex/econtexShortcuts}
\providecommand{\econtexBibMake}{./texmf-local/tex/latex/econtexBibMake}
\providecommand{\econtexBibStyle}{./texmf-local/bibtex/bst/econtex}
\providecommand{\notes}{./texmf-local/tex/latex/handout}
\providecommand{\handoutSetup}{./texmf-local/tex/latex/handoutSetup}
\providecommand{\handoutShortcuts}{./texmf-local/tex/latex/handoutShortcuts}
\providecommand{\handoutBibMake}{./texmf-local/tex/latex/handoutBibMake}
\providecommand{\handoutBibStyle}{./texmf-local/bibtex/bst/handout}

  

\documentclass{\econtex}
\usepackage{\econtexSetup}\usepackage{\econtexShortcuts}

\begin{document}\bibliographystyle{\econtexBibStyle}

\begin{verbatimwrite}{./AppendixSeveranceBody.tex}

%----------------------------------------------------------------------------------------
% Appendix 3 shows the derivation of incorporating severance into ctDiscrete framework. -- 1/30/2013
% It should be put right after the subsection:
% \subsection*{Stochastic Properties of Disposable Income and Saving for a Rainy Day}

\section*{Appendix 3: Extension of Derivation of Target Wealth to Include Unemployment Insurance}

The model described in \cite{ctDiscrete} assumes that income for
unemployed/retired households is zero.  A step in the direction of
realism is to recognize the existence (in most countries) of an unemployment
insurance system that guarantees some level of income to unemployed
persons.  The implications of such a system are straightforward to
model if we assume that the unemployment insurance benefit is a
constant proportion of the labor income earned in the first year of unemployment if not losing job.

In the perfect foresight context, receiving a constant payment with
perfect certainty is equivalent to receiving a lump sum ``severance''
payment whose value is equal to the PDV of the stream of future UI
payments.  Thus, for simplicity, we assume
$\SeverancePayment=\SeveranceRatio\cdot\labor\Wage$, which means
individuals will receive one-period severance payment $\SeverancePayment$ in the amount of
a certain ratio $\SeveranceRatio$ to labor income of the period when they first lose their jobs. After that, they will not receive any unemployment
insurance benefit.

The only modifications of the decision problem are to add
the severance payment and a corresponding lump-sum tax into the
dynamic budget constraint (DBC) of employed consumers in
\cite{ctDiscrete},
\begin{displaymath}
    \mLev_{t+1}=\left\{
    \begin{array}{ll}
    \bLev_{t+1}+\labor_{t+1}\Wage_{t+1}-\TaxUI_{t+1} & \mbox{w.p. } \urate\\
    \bLev_{t+1}+\SeverancePayment_{t+1} & \mbox{w.p. }1-\urate,
    \end{array}
    \right.
\end{displaymath}
where $\mLev$, $\bLev$ and $\labor\Wage$ denote market resources,
assets and labor income respectively. We let $\TaxUI=\urate\times
\SeverancePayment$ to ensure a balanced budget for the unemployment 
insurance system.\footnote{Each period, proportion
  $\urate$ of employed consumers lose their jobs, i.e., the ``exit
  rate'' in the current labor market is $\urate$.  In order to raise a
  corresponding amount of revenues, we need to assume that there is a
  ``birth rate'' of $\urate$ of new employed consumers, which means a
  same proportion of consumers are entering the labor market each
  period. Combined with the assumption that $\TaxUI=\urate\times
  \SeverancePayment$, the severance payment and the severance payment are
  balanced.} 

Following \cite{ctDiscrete}, we have the following condition derived from the Euler equation,
\begin{equation}
  1  = \PGro^{-\CRRA}\Rfree\Discount \left\{(1-\urate)\left(\frac{\cRatE_{t+1}}{\cRatE_{t}}\right)^{-\CRRA}+\urate \left(\frac{\cU_{t+1}}{\cRatE_{t}}\right)^{-\CRRA}\right\},
\end{equation}
where nonbold variables represent the bold variables normalized by
labor income $\labor\Wage$. Superscripts $e$ and $u$ represent the two
possible states.

To find the $\Delta c^e=0$ and $\Delta m^e=0$ loci, we let
$c_{t+1}^e=c_t^e\equiv c^e$ and $m_{t+1}^e=m_{t}^e\equiv m^e$. Given
$c_{t+1}^u=m_{t+1}^u\kappa^u$ ($\kappa^u$ is the MPC of an unemployed
consumer), combined with the modified DBC above, we have
\begin{equation*}
1 = \PGro^{-\CRRA}\Rfree\Discount \left\{(1-\urate)+\urate \left(\frac{\MPCU(\Rnorm(m^e-c^e)+\SeveranceRatio)}{\cRatE}\right)^{-\CRRA}\right\}.
\end{equation*}

Rearranging terms, the $\Delta c^e=0$ locus can be characterized as
\begin{equation}
\overbrace{\left(\frac{\PGro^{\CRRA}(\Rfree\Discount)^{-1}-\erate}{\urate}\right)^{1/\CRRA}}^{\straight}=\left(\frac{c^e}{(\Rnorm(m^e-c^e)+\SeveranceRatio)\MPCU}\right).
\end{equation}

Given the modified DBC of employed consumers, the $\Delta m^e=0$ locus becomes
\begin{equation}
m^e=\Rnorm(m^e-c^e)+(1-\urate\SeveranceRatio).
\end{equation}

Given the two equations above, we are able to obtain the exact formula for target wealth $\check{m}$, which is the steady state value of $m^e$. Following \cite{ctDiscrete}, define $\eta \equiv \Rnorm \MPCU \straight$. We have
\begin{eqnarray}
\frac{\eta\check{m}+\frac{\eta\SeveranceRatio}{\Rnorm}}{\eta+1}&=&(1-\Rnorm^{-1})\check{m}+\frac{1-\urate\SeveranceRatio}{\Rnorm}=\check{c}\nonumber\\
\left(\frac{1}{\Rnorm}-\frac{1}{\eta+1}\right)\check{m} &=& \frac{1}{\Rnorm}\left(1-\urate\SeveranceRatio-\frac{\eta\SeveranceRatio}{\eta+1}\right)\nonumber\\
\check{m}&=& \frac{(\eta+1)(1-\urate\SeveranceRatio)-\eta\SeveranceRatio}{\eta+1-\Rnorm}.
\end{eqnarray}

Clearly, target wealth decreases when the severance payment becomes more generous and it can even be negative if we make the severance ratio $\SeveranceRatio$ large enough.
%-----------------------------------------------------------------------------------------
\end{verbatimwrite}
\input ./AppendixSeveranceBody.tex

\bibliography{economics,AppendixSeverance,AppendixSeverance-Add}

\end{document}
