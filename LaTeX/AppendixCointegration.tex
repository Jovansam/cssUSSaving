\providecommand{\econtexRoot}{..}
% The \commands below are required to allow sharing of the same base code via Github between TeXLive on a local machine and ShareLaTeX.  This is an ugly solution to the requirement that custom LaTeX packages be accessible, and that ShareLaTeX seems to ignore symbolic links (even if they are relative links to valid locations)
\providecommand{\econtex}{./texmf-local/tex/latex/econtex}
\providecommand{\econtexSetup}{./texmf-local/tex/latex/econtexSetup}
\providecommand{\econtexShortcuts}{./texmf-local/tex/latex/econtexShortcuts}
\providecommand{\econtexBibMake}{./texmf-local/tex/latex/econtexBibMake}
\providecommand{\econtexBibStyle}{./texmf-local/bibtex/bst/econtex}
\providecommand{\notes}{./texmf-local/tex/latex/handout}
\providecommand{\handoutSetup}{./texmf-local/tex/latex/handoutSetup}
\providecommand{\handoutShortcuts}{./texmf-local/tex/latex/handoutShortcuts}
\providecommand{\handoutBibMake}{./texmf-local/tex/latex/handoutBibMake}
\providecommand{\handoutBibStyle}{./texmf-local/bibtex/bst/handout}

  

\documentclass{\econtex}
\usepackage{\econtexSetup}\usepackage{\econtexShortcuts}


\begin{document}

\begin{verbatimwrite}{./AppendixCointegrationBody.tex}

Our empirical modeling strategy differs deliberately from both the VAR
(and SVAR) literature and the approach based on an attempt to estimate 
a cointegrating relationship and then to analyze the dynamics of the residuals
from that relationship.  Although both of those literatures have made important
contributions, for our purposes they may obscure as much as they reveal 
about the key economic relationships we are interested in.

In this appendix, we begin by showing that, under the assumptions of our model,
a particular cointegrating relationship should in fact exist if all elements of 
the economic environment are perpetually unchanging.  However, in an economy 
with slow but continuous change in many of the `deep parameters' that determine
the cointegrating vector, serious errors of inference can be made by assuming that
the cointegrating vector is constant when it is not.  In particular, we show that
the usual procedure of first estimating a cointegrating vector (assumed to be constant 
over the sample), then analyzing the dynamics of its residuals, leads to serious errors
if the actual data are generated by our structural model.  The nature of the problems
of the cointegrating approach is not specific to our model, but generic.  Specifically,
the problem is that the time span of the available data is insufficient to reliably 
identify the cointegrating vector when there are omitted influences (like the 
increasing availability of credit over our sample period).  The problem is that in a 
finite sample, the first stage cointegrating regression inevitably will be biased 
in the presence of an important omitted variable, and that bias then contaminates
the residuals, imparting to them distorted dynamics that do not reflect the true
characteristics of the model.

We then show that, using the same data (generated by the model), our estimation methodology 
succeeds where the cointegration approach fails.  That is, our procedure is able to 
successfully uncover the `true' characteristics of the model generating the data 
from the simulated data from the model itself.  

\subsection{Cointegrating Vector Implied By the Model}

\cite{carroll:brookings} shows that a model of the class considered here, in which a 
target wealth-to-permanent-income ratio exists, will satisfy the equation

\begin{eqnarray}
  \sRat & \approx & \pGro (\mTarg-1)
\end{eqnarray}
where $\pGro$ is the `permanent' growth rate of income.  In the usual cointegrating methodology, the knowledge 
that such a cointegrating vector in principle exists would lead to the following approach to
understanding saving dynamics:
\begin{enumerate}
\item Estimate $\sRat_{t} = \alpha_{0} + \alpha_{1} (\mRat_{t}-1)$
\item Construct residuals $\epsilon_{t} = \sRat_{t} - \hat{\alpha}_{0} - \hat{\alpha}_{1} (\mRat_{t}-1)$
\item Analyze the dynamics of the residuals $\epsilon_{t}$ and the $\hat{\alpha}$ to discover answers to the questions 
of interest motivating the analysis.
\end{enumerate}

We have conducted exactly this exercise on the data generated by the estimated version of our model.  (The model predictions for
the saving rate that we use for this analysis are those depicted, for example, in Figure 12 in the paper).  

The first test of the success of the strategy is whether estimation of the cointegrating vector correctly uncovers the 
true cointegrating vector.  In this case, that vector is $\alpha_{0}=0$ and $\alpha_{1} = \pGro$.  In fact, the 
coefficient estimates are highly statistically significantly different from the `truth' that we know because we 
constructed the model ourselves.  % Jirka or Martin: Fill in

The second test of the success of the strategy is whether the dynamics of the $\epsilon_{t}$ residuals are a reasonable
approximation to the `true' dynamics of saving residuals in the model.  Some judgment is required to determine how 
to answer this question, but a plausible method would be to identify the `true' dynamics of the saving residual as reflecting the serial 
correlation of the saving rate predicted by the model over the sample period.  That is, again referring to figure 12, 
we can calculate the serial correlation of $\sRat_{t}-\hat{\sRat}_{t}$ where $\hat{\sRat}_{t}$ is the model-predicted saving rate.

Again, the results from the cointegrating approach do not resemble the
`truth' as interpreted in the
model.  % Jirka or Martin: fill in results.
As hinted above, the reason for this is that the first-stage
regression misidentifies the residuals because the sample size is
small relative to the time frame needed to reestablish the long-run
relationship between saving and wealth.

In contrast, when we apply our SMM estimation procedure to the data simulated by our model,
we correctly recover what we know are the `true' coefficient estimates (becuase they are the 
coefficient estimates used in simulating and constructing the data).  

It might be objected that this treatment sets up and then knocks down
a `straw man' version of the cointegrating approach; and indeed the
best of the practitioners of that approach (like
\cite{ducaEtAl10_creditArch} and Muellbauer in a number of papers)
have attempted to adjust the approach to account (for example) for
movements in credit availability over time.  It is possible that sufficiently
careful (and properly specified) cointegration approaches might remedy 
the deficiencies we articulate above.  

But our method has the appeal of great simplicity, and is at least able to 
reproduce the correct structural coefficients with confronted with data in 
which those coefficients are embedded, and in a context in which the standard
cointegration methodology fails.  Further research may uncover ways to integrate
our approach better with the cointegration methodology, but until that research
is conducted the answer is unclear (at least to us).  

\end{verbatimwrite}
\input ./AppendixCointegrationBody.tex

\bibliography{economics,AppendixCointegration,AppendixCointegration-Add}
%
%\write18{if [ `kpsewhich economics.bib` != '' ]; then touch economics.bib    ; fi} # This should be done only for final versions AFTER bibexport has occurred and \jobname.bib is populated 
\write18{if [ ! -f \jobname.bib     ]; then touch \jobname.bib     ; fi}
\write18{if [ ! -f \jobname-Add.bib ]; then touch \jobname-Add.bib ; fi}

 
\bibliography{economics,\jobname,\jobname-Add}


\end{document}
